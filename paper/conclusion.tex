\section{Conclusion and future work}
In this paper we have analyzed two recently proposed congestion control
protocols for RoCEv2 networks, namely DCQCN (ECN based) and TIMELY (delay
based), to both understand the behavior of the protocols in detail.
Using fluid models and control theoretic analysis
we derived stability regions for DCQCN, which demonstrated a somewhat odd
non-monotonic behavior of stability with respect to the number of contending
flows. We verified this behavior via packet level simulations. We also
demonstrated that DCQCN converges to a unique fixed point and that with the
proper choice of parameters the convergence is fast. In performing similar
analysis for TIMELY, we discovered that as proposed the TIMELY protocol has
infinite fixed points which could lead to unpredictable behavior and unbounded
unfairness. We provide a simple fix to TIMELY that makes the behavior similar to
DCQCN, with a unique operating point to which the protocol converges fast. 

Both protocols however exhibit a behavior that could be problematic in the
datacenter environment (with extremely low propagation delays), and that is the
operating queue length in both cases grows with the number of contending flows
and they could introduce significant latency. We showed that by using a PI
controller on the switch to mark packets, we can gurantee  we can guarantee
bounded delay and a fairness for DCQCN.  However we demonstrate and prove a
fundamental uncertainty result for TIMELY (and TIMELY-like protocols): if you
use delay as the only feedback signal for congestion control, then you can
either guarantee fairness or a bounded delay, but not both simultaneously. Based
on this reason, and the fact that ECN marking process on modern shared-buffer
switches eeffectively excludes queuing delay from feedback loop, we conclude
that ECN is a better congestion signal in data center environment. 

For future work, we are doing a full exploration of PI like controllers for
congestion control of RDMA in the datacenter, including a hardware
implementation. Our analysis also suggests that DCQCN can be simplified
considerably, to remove strange artifacts like the non-monotonic stability
behavior and that is the subject of our investigation as well.  We also plan to
extend our analysis to include the impact of PFC-induced PAUSES on the two
protocols.


%%% Local Variables:
%%% mode: latex
%%% TeX-master: "main"
%%% End:
