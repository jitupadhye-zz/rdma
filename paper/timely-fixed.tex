\subsection {Patched TIMELY}

\begin{figure}[t]
\center
\includegraphics[width=0.33\textwidth]{figures/timely_stability.eps}
\caption{Patched TIMELY stability}
\label{fig:timely_stability}
\end{figure}


In order to ensure the protocol has a unique fixed point, and all flows get fair share and are stable at the fixed point,
we make two minor modifications over TIMELY. As shown in Algorithm\ref{fig:timely_fixed_algo}, we only modified the last
three lines of Algorithm\ref{fig:timely_algo}.

\begin{algorithm}[t]
\footnotesize
{
\begin{algorithmic}[1]
%\Procedure{CalcRate}{$newRTT$}
\State $newRTTDiff \gets newRTT - prevRTT$
\State $prevRTT \gets newRTT$
\State $rttDiff \gets (1-\alpha) \cdot rttDiff + \alpha \cdot newRTTDiff$
\State $rttGradient = rttDiff/D_{minRTT}$
\If {$newRTT < T_{low}$}
        \State $rate \gets rate + \delta$
\ElsIf {$newRTT > T_{high}$}
        \State $rate \gets rate \cdot  (1 - \beta \cdot (1 - T_{high}/newRTT))$
\Else
		\State $weight \gets w(rttGradient)$	
		\State $error \gets \frac{{newRTT - RT{T_{ref}}}}{{RT{T_{ref}}}}$
        \State $rate \gets \delta (1-weight) +  rate \cdot (1 - \beta \cdot weight  \cdot error)$
\EndIf 
%\EndProcedure
\end{algorithmic}
}
\caption{TIMELY rate calculation}
\label{fig:timely_fixed_algo}
\end{algorithm}


First, we make the step of rate decrease rely on absolute queue length (RTT), instead of the gradient of queue length (RTT). 
This will ensure that there is only one fixed point for the queue length. In addition, all flow will have the same rate 
because the queue length, which is shared across flows, can uniquely determine the rate of each flow. This is 
similar to the use of ECN in DCQCN. The side effect is that, with different number of flows, the fixed point of 
queue will be different.

Second, we use a continuous weighting function $w(g)$ to make the transition between rate increase and rate decrease
smooth. This avoids the on-off behavior that causes oscillation, and stabilizes the protocol at the fixed point. 
This is similar to the fact that probabilistic ECN marking stabilizes TCP, QCN and DCQCN. With $w(g)$, we combine 
the two conditions of $g \le 0$ and $g>0$ in the $dR(t)/dt$ equation:


\begin{equation}
\small
\frac{{dR_i}}{{dt}} = \left\{ \begin{array}{ll}
\frac{\delta }{{\tau *}}, & q(t - \tau ') < C*{T_{low}}\\
\frac{{(1 - {w_i})\delta }}{{\tau *}} - \frac{{{w_i}\beta {R_i}(t)}}{{\tau *}}\frac{{q(t - \tau ') - q'}}{{q'}}, & Otherwise\\
 - \frac{\beta }{{\tau *}}(1 - \frac{{C*{T_{high}}}}{{q(t - \tau ')}})R_i(t), & q(t - \tau ') > C*{T_{high}}
\end{array} \right.\\
\label{eq:timely_fixed}
\end{equation}

where $w_i$, the weight of rate decreasing, is a function of $g_i$, and must satisfy $0 \le w_i(g_i) \le 1$ for any $g$. 
Intuitively, $w_i(g_i)$ is monotonically increasing with $g_i$, because larger RTT gradient should lead to larger 
weight of rate decreasing. In original TIMELY protocol, $w_i(g_i)$ is an indicator function of $g_i$, {\em i.e.,} 
$w_i(g_i)=1$ when $g \le 0$, and $w_i(g_i)=0$ when $g<0$. Here we simply use a linear function of $g_i$ for $w_i$:

\begin{equation}
\small
{w_i} = \left\{ \begin{array}{ll}
0, & {g_i \le  - \frac{1}{4}} \\
2{g_i} + \frac{1}{2}, & { - \frac{1}{4} < g_i < \frac{1}{4}} \\
1, & {g_i \ge \frac{1}{4}}
\end{array} \right.
\end{equation}

In Equation~\ref{eq:timely_fixed}, $q'$ is a reference queue length. We simply set it as $C*T_{low}$, 
so that we decrease the rate harsher if the queue length exceeds more than $C*T_{low}$. All the TIMELY
parameters remain the same except we set $\beta=0.008$ and $Seg=16KB$. 

\begin{thm}[Patched TIMELY's fixed point.]
The system described in Equation~\ref{eq:timely_fixed} has a unique fixed point.
\end{thm}
\begin{proof}
Let the LHS of Equation~\ref{eq:timely_g} be 0, and $q(t - \tau ') - q(t - \tau ' - \tau *) = 0$ at the fixed 
point, we know $g_i^*=0$. Thus, $w_i^*=0.5$. Let the LHS of Equation~\ref{eq:timely_fixed} be 0. Because
all flows share the same queue length $q^*$, we have:

\begin{equation}
\small
R_1^* = R_2^* = ... = R_N^*
\end{equation}

The sum of all flow rates must be $C$ at the fixed point, therefore each flow has fair share $C/N$. 
\end{proof}

In addition, we can easily obtain the fixed point of queue length:
\begin{equation}
\small
{q^*} = \frac{{N{R_{AI}}q'}}{{\beta C}} + q'
\label{eq:timely_fixed_q}
\end{equation}

\begin{thm}[Patched TIMELY convergence.]
The system described in Equation~\ref{eq:timely_fixed} always converges to the unique fixed point.
\end{thm}

Here we provide a brief proof. First, the queue length $q$ always
converges to the fixed point $q^*$. This can be proved by contradiction because whenever $q$ stabilizes
at $q>q^*$, we have ${\left. {\frac{{dR}}{{dt}}} \right|_{q > {q^*}}} < {\left. {\frac{{dR}}{{dt}}} \right|_{q = {q^*}}} = 0$, 
leading to queue length decrease. The case of $q<q^*$ is similar. 

Second, once queue length converges to a stable state, Equation~\ref{eq:timely_g} becomes
$\frac{{d{g_i}}}{{dt}} =  - \frac{\alpha }{{{\tau_i^*}}}{g_i}$. By turning it into a discrete model
with $\tau_i^*(t)$ as intervals, we see that $g_i$ converges to 0:

\begin{equation}
\small
\begin{array}{l}
\left| {{g_i}(n + 1)} \right| = \left( {1 - \frac{\alpha }{{\tau _i^*(n)}}} \right)\left| {{g_i}(n)} \right| = ...\\
 = \prod\limits_{t = 0}^n {\left( {1 - \frac{\alpha }{{{\tau ^*}(t)}}} \right)} \left| {{g_i}(0)} \right| \le {\left( {1 - \frac{\alpha }{{{D_{\min RTT}}}}} \right)^n}\left| {{g_i}(0)} \right|
\end{array}
\end{equation}

Finally, after $g_i$ converges to 0, which means $w_i=0.5$, we rewrite the $j$th flow's Equation~\ref{eq:timely_fixed} 
and subtract it from $i$th flow's. Without losing generality, we choose $i,j, s.t.$ $i$th flow is the fatest among all flows,
whereas $j$th flow is the slowest. After simplification, we get:

\begin{equation}
\small
\frac{{d\left( {{R_i} - {R_j}} \right)}}{{dt}} = \frac{\delta }{{2Seg}}\left( {1 - \frac{N}{C}\left( {{R_i} + {R_j}} \right)} \right)\left( {{R_i} - {R_j}} \right)
\end{equation}

Among the total $N$ flows, there must exist at least one flow not slower than the fair share $C/N$.
Therefore, ${{R_i} + {R_j}} > C/N$ since $R_i$, the fatest flow, is not slower than $C/N$.
Solving the differential equation with ${{R_i} + {R_j}}$ as the variable, we get that the rate 
different between the fatest and slowest flow decrease over time exponentially, as we shall see below. 

\para{Convergence speed.} We estimate the TIMELY convergence time {\em after}
the queue stabilizes. We start from the case of $N=2$, where $R_i + R_j = C$ because of Equation~\ref{timely_q}.
Then we get:

\begin{equation}
\small
{R_i}(t) - {R_j}(t) = \left( {{R_i}(0) - {R_j}(0)} \right){e^{\frac{{\delta (1 - N)}}{{2Seg}}t}}
\end{equation}

We use the same definition of ``converged'' as in Section~\ref{sec:dcqcn_convergence}, then the convergence
time is the smallest $t$ that satisfies (assuming two flows start with maximum possible rate difference, $C$):

\begin{equation}
\small
{e^{\frac{{\delta (1 - N)}}{{2Seg}}t}}C \le \frac{{0.1 \times C}}{N}
\end{equation}

With $N=2$ and parameters we choose earlier, we get $t = 76.7ms$. With this equation, it is easy to see that
with larger $N$,\footnote{Details omitted.} larger $\delta$ or smaller $Seg$, patched TIMELY will converge faster.

We verify patched TIMELY convergence and stability using simulations. As shown in Figure TODO, flows with different 
initial rates converge to the fixed point and are stable without oscillation. 

\para{Stability.} We use the same technique as analyzing
DCQCN stability, {\em i.e.,} linearize the equations, Laplace transform and compute the phase margin 
of its characteristic equation. The phase margin result shows this system is stable when the number of 
flows is less than 40 (Figure~\ref{fig:timely_stability}). After 40 flows, the phase margin falls below 
0 rapidly because more flows lead to larger queue size (see Equation~\ref{eq:timely_fixed_q}), 
thus leading to larger feedback delay (see Equation~\ref{eq:timely_taup}). 
This leads to system instability. We conclude that with some minor tuning, TIMELY can be stable 
within a range of number of flows.