TIMELY~\cite{timely} is an end-to-end, rate-based congestion control algorithm
that uses changes in RTT as a congestion signal.  It relies on NIC hardware to
ontain fine-grained RTT measurements. RTT is estimated once per completion
event~\cite{rocev2}, which signals the successful transmission of a chunk
(16-64KB) of packets. Upon receiving a new RTT sample, TIMELY computes new rate
for the flow, as shown in Algorithm \ref{fig:timely_algo}.  If the new RTT
sample is less than ($T_{low}$), TIMELY increases send rate additively by
$\delta$. If the new sample is more than ($T_{high}$), rate is decreased
multiplicatively by $\beta$. If the new sample is between $T_{low}$ and
$T_{high}$, the rate change depends on the RTT gradient. The gradient is defined
as the normalized change between two successive RTT samples. If the gradident is
positive (i.e. RTT is increasing), sending rate is reduced multiplicatively, in
proportion to the RTT gradient.  Otherwise, it is increased additively by
$\delta$.

TIMELY flows do not start at line rate. If there are N active flows at a sender,
a new flow starts at rate C/(N+1), where C is the interface link
bandwidth~\cite{timely}.

\begin{algorithm}[t]
\footnotesize
{
\begin{algorithmic}[1]
%\Procedure{CalcRate}{$newRTT$}
\State $newRTTDiff \gets newRTT - prevRTT$
\State $prevRTT \gets newRTT$
\State $rttDiff \gets (1-\alpha) \cdot rttDiff + \alpha \cdot newRTTDiff$
\State $rttGradient = rttDiff/D_{minRTT}$
\If {$newRTT < T_{low}$}
        \State $rate \gets rate + \delta$
\ElsIf {$newRTT > T_{high}$}
        \State $rate \gets rate \cdot  (1 - \beta \cdot (1 - T_{high}/newRTT))$
\ElsIf {$rttGradiant \le 0$}
        \State $rate \gets rate + \delta$
\Else
        \State $rate \gets rate \cdot (1 - \beta \cdot rttGradient)$
\EndIf 
%\EndProcedure
\end{algorithmic}
}
\caption{TIMELY rate calculation}
\label{fig:timely_algo}
\end{algorithm}
