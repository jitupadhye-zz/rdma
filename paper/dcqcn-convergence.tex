\subsection{Convergence}
\begin{figure}[t]
\center
\includegraphics[width=0.4\textwidth]{figures/dcqcn_convergence.eps}
\caption{DCQCN flow rate update.}
\label{fig:dcqcn_convergence}
\end{figure}

In Section~\ref{sec:dcqcn_stability}, we showed that for reasonable parameter
settings,  DCQCN is stable {\em after} flows converge to the unique fixed point.
However, two questions remain unanswered: $(i)$ do flows always converge to the
fair share, and $(ii)$ how fast do flows converge? 

We cannot answer these questions using the fluid model, so we construct and
analyze a discrete model of the rate adjustment at the RP. The default parameter
settings given in~\cite{dcqcn} set both the Timer $T$ and $\alpha$ update
interval $\tau '$ equal to $55\mu s$. Thus, we use $\tau '$ as the unit of time.
The process of DCQCN rate update is similar to TCP AIMD, as shown in
Figure~\ref{fig:dcqcn_convergence}.  The flows get the peak rates at $T_k$. For
simplicity, we assume all flows get to the peak at the same time, which is
likely to happen because the queue peak at the bottleneck leads to high marking
probability for all flows. When the flow gets ECN marks at $T_k$, it reduces its
rate in one unit of time, then starts multiple consecutive additive rate
increases on $R_T^{(i)}$:\footnote{For simplicity, we omit hyper-increase and
fast-recover phases, which shall help flows converge even faster.}

\begin{equation}
\small
R_T^{(i)}({T_{k + 1}}) = \left( {1 - \frac{{{\alpha ^{(i)}}({T_k})}}{2}} \right)R_C^{(i)}({T_k}) + \left( {\Delta {T_k} - 1} \right){R_{AI}}
\label{eq:converge_rc}
\end{equation}
where $\Delta {T_k} \buildrel \Delta \over = {T_{k + 1}} - {T_k}$. During the consecutive 
additive rate increase, $R_C^{(i)}$ and $R_T^{(i)}$ have following relationship according 
to DCQCN's definition:
\begin{equation}
\small
\begin{array}{l}
R_C^{(i)}(t + 1) = \frac{1}{2}\left( {R_C^{(i)}(t) + R_T^{(i)}(t + 1)} \right)\\
R_T^{(i)}(t + 1) = R_T^{(i)}(t) + {R_{AI}}
\end{array}
\end{equation}

Subtract one of the above by the other one, we get:

\begin{equation}
\small
\begin{array}{l}
R_C^{(i)}(t + n) - R_T^{(i)}(t + n) + {R_{AI}}\\
 = \frac{1}{2}\left( {R_C^{(i)}(t + n - 1) - R_T^{(i)}(t + n - 1) + {R_{AI}}} \right)\\
 = ... = {\left( {\frac{1}{2}} \right)^n}\left( {R_C^{(i)}(t) - R_T^{(i)}(t) + {R_{AI}}} \right)
\end{array}
\end{equation}

From this, we know that with a consecutive additive rate increase phase, $R_T^{(i)} - R_C^{(i)}$
will converge towards $R_{AI}$ exponentially. In common cases, the step of additive increase
$\Delta {T_k} - 1$ is easily larger than 10, sometimes even 100, as we shall see soon. Therefore,
we can safely estimate $R_T^{(i)}$ by:

\begin{equation}
\small
R_T^{(i)}({T_k}) \approx R_C^{(i)}({T_k}) + {R_{AI}},\forall k = 1,2,...
\end{equation}

Back to Equation~\ref{eq:converge_rc}, we must know $\alpha ^{(i)}({T_k})$ and $\Delta T_k$ to 
further analyze the flow rates. During $\Delta T_k$, $\alpha ^{(i)}({T_k})$ has one increase event
and $\Delta T_k - 1$ decrease events, as defined by DCQCN:

\begin{equation}
\small
{\alpha ^{(i)}}({T_{k + 1}}) = {(1 - g)^{\Delta {T_k} - 1}}\left( {(1 - g){\alpha ^{(i)}}({T_k}) + g} \right)
\label{eq:converge_alpha}
\end{equation}

For another flow, {\em e.g.,} the $j$th flow, we can simply rewrite above equation by replacing 
$(i)$ with $(j)$. We subtract the equation of $j$th flow from the equation of $i$th flow, we get:

\begin{equation}
\small
\begin{array}{l}
{\alpha ^{(i)}}({T_{k + 1}}) - {\alpha ^{(j)}}({T_{k + 1}}) = {(1 - g)^{\Delta {T_k}}}\left( {{\alpha ^{(i)}}({T_k}) - {\alpha ^{(j)}}({T_k})} \right)\\
 = ... = {(1 - g)^{\sum\nolimits_{l = 0}^k {\Delta {T_l}} }}\left( {{\alpha ^{(i)}}({T_0}) - {\alpha ^{(j)}}({T_0})} \right)
\end{array}
\end{equation}

This tells us the difference of $\alpha^{(i)}$ of different flows will quickly decrease exponentially. So $\alpha^{(i)}$
of different flows will converge to the same value, and the converging speed is determined by $g$ and
$\Delta T_k$. 

Once the $\alpha$ of different flows converge to the same value $\alpha(T_{k'})$ at $T_{k'}$, we can show 
the rates $R_C$ converge afterwards. We rewrite the any $j$th flow's Equation~\ref{eq:converge_rc}, and subtract it 
from Equation~\ref{eq:converge_rc}, we get:

\begin{equation}
\small
\begin{array}{l}
R_C^{(i)}({T_{k + 1}}) - R_C^{(j)}({T_{k + 1}}) = \left( {1 - \frac{{\alpha ({T_k})}}{2}} \right)\left( {R_C^{(i)}({T_k}) - R_C^{(j)}({T_k})} \right)\\
 = ... = \prod\limits_{l = k'}^k {\left( {1 - \frac{{\alpha ({T_l})}}{2}} \right)} \left( {R_C^{(i)}({T_{k'}}) - R_C^{(j)}({T_{k'}})} \right)
\end{array}
\label{eq:converge}
\end{equation}

As long as $\alpha ({T_k})$ has a lower bound that is greater than 0, the rates $R_C$ of different flows 
converge exponentially. Before we prove this, we need to have an estimation of $\Delta T_k$ using $\alpha$.
In the period of $\Delta T_k$, after the first time unit of rate decreasing, the aggregated flow rates
will climb back to $R_T(T_{k+1})$ by $NR_{AI}$ every time unit. Thus we have:

\begin{equation}
\small\Delta {T_k} = 1 + \frac{{\sum\limits_{i = 1}^N {\left( {R_T^{(i)}({T_{k + 1}}) - \left( {1 - {\alpha ^{(i)}}({T_k})/2} \right)R_C^{(i)}({T_k})} \right)} }}{{N{R_{AI}}}}
\end{equation}

Suppose $\alpha^{(i)}({T_k})$ already converged to the same value $\alpha ({T_k})$, we can largely simplify it:

\begin{equation}
\small
\begin{array}{l}
\Delta {T_k} = 1 + \frac{{\sum\limits_{i = 1}^N {R_T^{(i)}({T_{k + 1}})}  - \left( {1 - \alpha ({T_k})/2} \right)\sum\limits_{i = 1}^N {R_C^{(i)}({T_k})} }}{{N{R_{AI}}}}\\
 \approx 1 + \frac{{\left( {C + tN{R_{AI}} + N{R_{AI}}} \right) - \left( {1 - \alpha ({T_k})/2} \right)\left( {C + tN{R_{AI}}} \right)}}{{N{R_{AI}}}}\\
 = 2 + \left( {\frac{t}{2} + \frac{C}{{2N{R_{AI}}}}} \right)\alpha ({T_k})
\end{array}
\label{eq:converge_tk}
\end{equation}

Where $t$ is the time it takes for the flows to build up queue and get packets ECN-marked, after
the aggregated flow rates exceed link capacity $C$, as shown in Figure~\ref{fig:dcqcn_convergence}.
We can estimate $t$ by the queue being built up:

\begin{equation}
\small
\begin{array}{l}
N\tau '\left( {{R_{AI}} + 2{R_{AI}} + ... + t{R_{AI}}} \right) = {Q_{ECN}} \le {K_{\max }}\\
 \Rightarrow t \le \left( { - 1 + \sqrt {1 + \frac{{8{K_{\max }}}}{{N{R_{AI}}\tau '}}} } \right)/2
\end{array}
\end{equation}

Combining Equation~\ref{eq:converge_alpha} and~\ref{eq:converge_tk}, we denote the fixed point of 
the $\alpha$ array as $\alpha^{*}$ (and corresponding $\Delta T^{*}$ from Equation~\ref{eq:converge_tk}):

\begin{equation}
\small
{\alpha ^*} = {(1 - g)^{\Delta {T^*}}}\left( {(1 - g){\alpha ^*} + g} \right)
\end{equation}

Next, we will prove:

\begin{equation}
\small
\alpha ({T_0}) > ... > \alpha ({T_k}) > \alpha ({T_{k + 1}}) > ... > {\alpha ^*} > 0
\end{equation}

Once this is proved, $\alpha ({T_k})$ has a non-zero lower bound, $R_C$ will converge exponentially.
We prove this by mathematical induction. First of all, the initial value of $\alpha$ is 1, as defined
by DCQCN. So, $\alpha ({T_0}) > {\alpha ^*} > 0$. Now assuming $\alpha ({T_k}) > {\alpha ^*} > 0$, we 
prove $\alpha ({T_k}) > \alpha ({T_{k+1}}) > {\alpha ^*}$. We define $f(\alpha)$ as the RHS of 
Equation~\ref{eq:converge_alpha}:

\begin{equation}
\small
f(\alpha ) = {(1 - g)^{2 + \left( {\frac{t}{2} + \frac{C}{{2N{R_{AI}}}}} \right)\alpha }}\left( {(1 - g)\alpha  + g} \right)
\end{equation}

By analyzing the derivative of $f(\alpha )$, it is not hard to see that with common parameter 
settings, $f(\alpha )$ is monotonically increasing. Therefore, 

\begin{equation}
\small
\alpha ({T_{k + 1}}) = f\left( {\alpha ({T_k})} \right) > f\left( {{\alpha ^*}} \right) = {\alpha ^*}
\end{equation}

In addition, because $\alpha ({T_k}) > {\alpha ^*} => \Delta {T_k} > \Delta {T^*}$, 
$\alpha^{*}$ satisfies:

\begin{equation}
\small
{\alpha ^*} > {(1 - g)^{\Delta {T_k}}}\left( {(1 - g){\alpha ^*} + g} \right)
\end{equation}

Subtract this from Equation~\ref{eq:converge_alpha}, we see $\alpha ({T_k})$ is exponentially 
converging to $\alpha ^*$:

\begin{equation}
\small
0 < \alpha ({T_{k + 1}}) - {\alpha ^*} < {(1 - g)^{\Delta {T_n}}}\left( {\alpha ({T_k}) - {\alpha ^*}} \right)
\end{equation}

This concludes our proof. Combining Equation~\ref{eq:converge_rc}, we know that $R_C$ is 
converging at the rate of at least $( {1 - \frac{{{\alpha ^{*}}}}{2}} )^k$.

\begin{figure}[t]
\center
\includegraphics[width=0.4\textwidth]{figures/dcqcn_convergence_time.eps}
\caption{Upper bound on DCQCN convergence time under conservative assumptions}
\label{fig:dcqcn_convergence_time}
\end{figure}

\para{Convergence time.} We now derive an upper bound of DCQCN's flow
convergence time under certain conservative assumptions. 

In the above analysis, the convergence has two phases: 1) $\alpha ^{(i)}$
converge to the same value (not necessarily at $\alpha^*$); 2) $R_C^{(i)}$
converge along with $\alpha \to \alpha^*$.  In practice, the flows that
encounter congestion for the first time all have the same initial value of 1.
Therefore, we mainly focus on the second phase. We define the system as having
``converged'' when the difference of any two flows is no more than 10\% of
expected fair share.

We now derive an upper bound convergence time of under several worst-case
assumptions. First, we assume that the value of $\alpha ({T_k})$ is $\alpha ^*$.
In practice, $\alpha ({T_k}) > \alpha ^*$, so the flows will converge faster
than our estimation, given the term $(1 - \alpha /2)^T$ in
Equation~\ref{eq:converge}.  Second, we assume that the switch does not mark ECN
until the queue length hits $K_{max}$, opposed to using RED. 

The maximum possible difference between initial rates of two flows can at most
be $C$ (i.e. one flow starts at 0, other flow starts at line rate). So, an upper bound on the
convergence time is the time required for this difference to fall from $C$ to
less than $\frac{0.1 \times C}{N}$. This is simply the smallest value of $x$
that satisfies the following inequality: 

\begin{equation} 
\small 
{ \left( {1 - \frac{{{\alpha ^*}}}{2}} \right)^{\frac{x}{{\tau '\Delta{T^*}}}}}C \le \frac{0.1 \times C}{N} 
\end{equation}

We solve equation numerically for different parameter settings. The results are
shown Figure~\ref{fig:dcqcn_convergence_time}. With default parameter settings,
DCQCN converges within 100ms, and the convergenced time decreases quickly as the
number of flow increases.  We have also calculated convergence time for a range
of different values of $R_{AI}$, $K_{max}$ and $g$. Sample results are shown in
Figure~\ref{fig:dcqcn_converge_time}. We found that convergence time is
determined primarily $R_{AI}$, which is not surprising.

%Note that Figure~\ref{fig:dcqcn_converge_time}
%is the upper bound of convergence time with conservative assumptions. 
