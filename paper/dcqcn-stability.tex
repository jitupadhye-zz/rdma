\subsection{Stability}
\label{sec:dcqcn_stability}

\begin{figure*}[t]
\center
\subfigure[Default parameters ($R_{AI}=40Mbps$, $K_{max}=200KB$).]
{
\includegraphics[width=0.3\textwidth]{figures/dcqcn_stability.eps}
\label{fig:dcqcn_stability_default}
}
\subfigure[$R_{AI}=10Mbps$.]
{
\includegraphics[width=0.3\textwidth]{figures/dcqcn_stability_rai.eps}
\label{fig:dcqcn_stability_rai}
}
\subfigure[$R_{AI}=10Mbps$ and $K_{max}=1000KB$.]
{
\includegraphics[width=0.3\textwidth]{figures/dcqcn_stability_rai_kmax.eps}
\label{fig:dcqcn_stability_rai_kmax}
}
\vspace{-1em}
\caption{DCQCN stability}
\vspace{-1em}
\label{fig:dcqcn_stability}
\end{figure*}
We first obtain the fixed point of the system, and linearize the model around
the fixed point. We analyze the linearized model for stability using standard
frequency domain techniques~\cite{controltheory}.  

\begin{thm}[DCQCN's unique fixed point]
A unique fixed point of queue length exists for DCQCN. 
\end{thm}
\begin{proof}
By setting the left-hand side of Equation~\ref{eq:q} to 0,
we see that any fixed points of the DCQCN (if they exist) must
satisfy:
\begin{equation}
\small
\sum\limits_{i = 1}^N {R_C^{(i)}(t)} = C
\label{eq:fixedrc}
\end{equation}
At any of the fixed points, we assume the value of $p$ is $p^*$, which is shared
by all flows. The queue length and per-flow $\alpha^{(i)}$ at the fixed points
are determined by Equation~\ref{eq:mark} and \ref{eq:alpha}:
\begin{equation}
\small
{q^*} = \frac{{{p^*}}}{{{p_{max}}}}\left( {{K_{max}} - {K_{min}}} \right) + {K_{min}}
\end{equation}
\begin{equation}
\small
\alpha^{(i)*}  = 1 - {(1 - p^*)^{{\tau '}R_C^{(i)*}}}
\end{equation}
Next, we show that $p^*$ exists and is uniquely determined by $R_C^{(i)*}$ in
the DCQCN model. Combining Equation~\ref{eq:rt} and \ref{eq:rc}, 
we eliminate the variable $R_T^{(i)*}$. After simplification, we see that the value 
of $p^*$ is determined by:
\begin{equation}
\small
\frac{{{a^2}\alpha^{(i)*} }}{{(b + d)(c + e)}} = {\tau ^2}{R_{AI}}R_C^{(i)*}
\label{eq:p_fixed}
\end{equation}
Where we denote $a, b, c, d, e$ as follows:
\begin{equation}
\small
\begin{array}{l}
a = 1 - {(1 - p^*)^{\tau {R_C^{(i)*}}}},b = \frac{{p^*}}{{{{(1 - p^*)}^{ - B}} - 1}},c = \frac{{{{(1 - p^*)}^{FB}}p^*}}{{{{(1 - p^*)}^{ - B}} - 1}},\\
d = \frac{{p^*}}{{{{(1 - p^*)}^{ - T{R_C^{(i)*}}}} - 1}},e = \frac{{{{(1 - p^*)}^{FT{R_C^{(i)*}}}}p^*}}{{{{(1 - p^*)}^{ - T{R_C^{(i)*}}}} - 1}}
\end{array}
\end{equation}
%at the fixed point,
%we get the two forms of $R_T^{(i)*}$, respectively:
%\begin{equation}
%\small
%{R_T^{(i)*}} = R_C^{(i)*} + \frac{{a\alpha^{(i)*} }}{{(b + d)\tau }}
%\end{equation}
%\begin{equation}
%\small
%{R_T^{(i)*}} = R_C^{(i)*}\left( {1 + \frac{{(c + e)\tau {R_{AI}}}}{a}} \right)
%\end{equation}
%Combining the two forms of $R_T^{(i)*}$, we see that the value of $p^*$ is determined by:
The LHS of Equation (\ref{eq:p_fixed}) is a monotonic function of $p$ when $p \in [0,1]$.
Furthermore, when $p = 0$, the LHS is smaller than RHS, and vice versa when $p =
1$. Thus DCQCN has a unique fixed point of marking probability $p^*$, leading to
a unique fixed point of queue length $q^*$.
\end{proof}

Next we approximate the value of $p^*$. Numerical analysis shows that $p^*$ is 
typically very close to 0. Therefore, we approximate the LHS using Taylor series around $p=0$.
%\begin{equation}
%\small
%\frac{{{a^2}\alpha }}{{(b + d)(c + e)}} = \frac{{(R_C^{(i)*})^3{\tau ^2}\tau '}}{{{{\left( {\frac{1}{B} + \frac{1}{{TR_C^{(i)*}}}} \right)}^2}}}{p^3} + O\left( {{p^4}} \right)
%\end{equation}
After omitting the $O(p^4)$ term in the Taylor series, we have:
\begin{equation}
\small
{p^*} = \sqrt[3]{{\frac{{{R_{AI}}}}{{{\tau '}(R_C^{(i)*})^2}}{{\left( {\frac{1}{B} + \frac{1}{TR_C^{(i)*}}} \right)}^2}}}
\label{eq:fixedp}
\end{equation}
With this estimation, we see that $p^*$ uniquely determine $R_C^{(i)*}$. 
Since $p^*$ is shared by all flows $i$, $i = 1, 2, ..., N$, we have:\footnote{In 
Section~\ref{sec:dcqcn_convergence}, we rigorously prove that all flows
converge to the same rate.}
\begin{equation}
\small
R_C^{(1)*} = R_C^{(2)*} = ... = R_C^{(N)*}
\end{equation}
Combining this with Equation~\ref{eq:fixedrc}, proves that at the fixed point, $R_C^{(i)*} = \frac{C}{N}$, 
$i = 1, 2, ..., N$.

\para{Stability analysis.} 
We test the system against {\em Bode Stability Criteria}~\cite{controltheory}. 
Results are shown in Figure~\ref{fig:dcqcn_stability}. See~\cite{fullpaper} for
details of the derivation of the characteristic equation for $R_C$.
The degree of stability is shown as {\em
Phase Margin}. The system is stable when its {\em Phase Margin} is larger than
0, and the larger {\em Phase Margin} means the system is more stable.

We analyze DCQCN stability in different conditions, particularly with different
control signal delays (propagation delay plus queuing delay in practice), and
different number of flows. An ideal protocol should be tolerant with large delay
and scalable to any number of flows. As Figure~\ref{fig:dcqcn_stability} shows,
DCQCN, with default parameters, is mostly stable. 

\begin{figure*}[t]
\centering
\mbox{
\begin{minipage}{0.66\textwidth}
\subfigure[] {\includegraphics[width=0.49\columnwidth]{figures/stable_rate_4.pdf}}
\subfigure[] {\includegraphics[width=0.49\columnwidth]{figures/stable_q_4.pdf}}
\subfigure[] {\includegraphics[width=0.49\columnwidth]{figures/stable_rate_85.pdf}}
\subfigure[] {\includegraphics[width=0.49\columnwidth]{figures/stable_q_85.pdf}}
\vspace{-0.5em}
\caption{Impact of delay and number of flows on DCQCN stability}
\vspace{-0.5em}
\label{fig:dcqcn_unstable}
\end{minipage}

\begin{minipage}{0.33\textwidth}
\includegraphics[width=0.99\columnwidth]{figures/stable_queue_85_ns.pdf}
\vspace{-2em}
\caption{NS simulations confirm lack of stability}
\label{fig:dcqcn_unstable_ns}

\includegraphics[width=0.99\columnwidth]{figures/dcqcn_stability_100gbps.eps}
\vspace{-2em}
\caption{DCQCN stability with 100Gbps}
\vspace{-0.5em}
\label{fig:dcqcn_100gbps}
\end{minipage}
}
\end{figure*}

%\begin{figure}[t]
%\center
%\includegraphics[width=0.4\textwidth]{figures/stable_queue_85_ns.pdf}
%\caption{NS simulations confirm lack of stability}
%\label{fig:dcqcn_unstable_ns}
%\end{figure}

However, unlike TCP~\cite{misra2000fluid}, the relationship between number of
flows and the phase margin is non-monotonic. When the delay is large, {\em e.g.,
100$\mu$s}, the phase margin dips below zero for certain number of flows, before
rising again.  For the set of parameters we have chosen, the system can be
unstable with 10 flows at high feedback delays.  DCQCN is increasingly stable
with larger number of flows, which means good scalability. This point is further
illustrated in the fluid model results shown Figure~\ref{fig:dcqcn_unstable}.
When the feedback delay is small (4 $\mu$s), DCQCN is stable - flow rates, and
queue length quickly\footnote{Remember that DCQCN flows always start at line
rate.} stabilizes regardless of the number of flows. However, when the delay is
large (85$\mu$s), the protocol is unstable for 10 flows. It is, however, stable
for 2 and 64 flows. Figure~\ref{fig:dcqcn_unstable_ns} shows the instability
with packet-level simulations.

While this problem may not be particularly serious in practice, it can be easily
fixed by tuning the values of $R_{AI}$ and $K_{max}$.  Smaller $R_{AI}$ means
flows increase their rate more gently, and stabilizes the system.  Similarly,
larger $K_{max} - K_{min}$ makes rate decreasing more fine grained, because the
perturbation of queue length leads to smaller perturbation in marking
probability.  We show these trends in Figures \ref{fig:dcqcn_stability_rai} and
\ref{fig:dcqcn_stability_rai_kmax}.  With small $R_{AI}$ and large $K_{max}$,
DCQCN can be always stable even when the control signal delay reaches 100$\mu
s$, which equals to the propagation delay of a $30KM$ cable, or $500KB$ queuing
delay. Such large delays are rare in modern datacenter networks. 

Note that tuning $R_{AI}$ and $K_{max}$ is a trade-off between stability and
latency. Smaller $R_{AI}$ leads to slower ramp-up, while larger $K_{max}$ leads
to larger queue length. In most cases, the default parameters strike a good
enough balance between stability and latency.

\para{Stability at 100Gbps.}  As the datacenter network fabric is moving towards
100Gbps bandwidth, we analyze DCQCN stability for $C=100Gbps$.
Figure~\ref{fig:dcqcn_100gbps} shows DCQCN is stable with default parameter when
the delay is small, {\em e.g.,} close to $0\mu s$. However, at higher delays
($50-100 \mu s$), the phase margin is much lower than it is for 40Gbps links, and
the system is unstable unless $N$ is high. While we can stabilize the system by
tuning down $R_{AI}$ and tuning up $K_{max}$, it hurts convergence speed and
queue length significantly. In \S~\ref{sec:discuss} we argue that using a PI
controller is a more principled approach to enhance stability.
