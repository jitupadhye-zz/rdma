\begin{abstract}
Large scale RDMA deployments require congestion control. Recently, two
protocols, namely, DCQCN and TIMELY have been proposed for thus purpose. In this
paper, using fluid models, control theory and simulations, we study the
stability and convergence properties of these two protocols. Apart from
derivation of stability margins, and convergence bounds, our analysis uncovers
several surprising behaviors of these protocols. For example, we show that DCQCN
exhibits non-monotonic stability behavior, and that TIMELY can converge to
stable regime with arbitrary unfairness.  We propose simple fixes that can
alleviate these problems. More generally, we show that when properly tuned, the
performance of the two protocols is comparable, and thus, the decision about
which protocol to use can largely be made on the basis of engineering
considerations.
\end{abstract}
