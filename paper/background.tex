\section{Background}
The Remote Direct Memory Access (RDMA) technology offers high throughput(40Gbps
or more), low latency (few $\mu$s), and low CPU overhead (1-2\%), by bypassing
the end-host kernels during data transfer. Instead, network interface cards
(NICs) transfer data in and out of pre-registered memory buffers at the two end
hosts. The networking protocol is implemented entirely on the NICs.

Modern data center networks deploy RDMA using the RDMA over Converged Ethernet
V2 (RoCEv2)~\cite{rocev2} standard.  RoCEv2 requires a lossless (or, more
accurately, drop-free) L2 layer. At the Ethernet layer, this is achieved using
Priority Flow Control (PFC). PFC prevents buffer overflow on Ethernet switches
and NICs, as follows. The switches and NICs track ingress queues. When the queue
exceeds a certain threshold, a PAUSE message is sent to the upstream entity. The
uplink entity then stops sending on that link till it gets an RESUME message.
PFC is a blunt mechanism, since it does not operate on a per-flow basis. This
leads to several well known problems such as head-of-the-line blocking, and
victimization~\cite{dcqcn,tcp-bolt}. 

To alleviate these problems, per-flow, end-to-end congestion control is
necessary.  Recently, two protocols, DCQCN~\cite{dcqcn} and TIMELY~\cite{timely}
were proposed for this purpose. Since PFC prevents packet drops due to buffer
overflow, either ECN or increase in RTT are the only two available congestion
signals.  DCQCN uses ECN marking as the congestion signal, while TIMELY uses
changes to RTT.

