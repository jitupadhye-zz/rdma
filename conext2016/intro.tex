\section{Introduction}

Large cloud service providers are turning to Remote DMA (RDMA) technology to
support their most demanding applications~\cite{timely,tcp-bolt,dcqcn}.  RDMA
offers significantly higher bandwidth and lower latency than the traditional
TCP/IP stack, while minimizing CPU overhead~\cite{farm,timely,dcqcn}. 

Today, RDMA is deployed using the RoCEv2 standard~\cite{rocev2}. To ensure
efficient operation, RoCEv2 uses Priority Flow Control (PFC)~\cite{pfc} to
prevent packet drops due to buffer overflow.  Since there are no packet drops,
any end-to-end congestion control protocol for RoCEv2 networks must use either
ECN markings, or delay as the congestion signal.  Last year, two protocols were
proposed for this purpose, namely DCQCN~\cite{dcqcn} and TIMELY~\cite{timely}.
DCQCN uses ECN marking as a congestion signal, while TIMELY measures changes to
end-to-end delay.

% what we do.

In this paper, we analyze DCQCN and TIMELY for stability, convergence,
fairness and flow completion time.

We have two motivations for this work. First, we want to understand the
performance of DCQCN and TIMELY in detail. Given their potential for widespread
deployment, we want to understand the tradeoffs made by the two protocols in
detail, as well as offer guidance for parameter tuning.  Second, using insights
drawn from the analysis, we want to answer a broader question: are there
fundamental reasons to prefer either ECN or delay as the congestion signal in
data center networks? 

% Methodology.

To this end, we analyze DCQCN and TIMELY using fluid models and NS3~\cite{NS3}
simulations. Fluid models are useful for analyzing properties such as stability,
and for rapid exploration of parameter space. However, fluid models cannot
tractably model all features of complex protocols like DCQCN and TIMELY. Nor can
fluid models compute measures like flow completion times. Thus, we also study
the protocols using detailed, packet-level simulations using NS3. Our
simulations in NS3 implement all known features of the protocols. We will publish
our NS3 and MATLAB code.

% summary of contributions and findings.

Before we proceed, we want to stress two things.  First, we limit our discussion
to ECN and delay as congestion signals, since they are supported by current
hardware. Protocols that require switches to do
more~\cite{pfabric,rcp,katabi2002congestion} or use a central
controller~\cite{perry2014fastpass,deadline} are outside the scope of this
paper. We also do not consider designing a new congestion control protocol that
would use both ECN and delay as congestion signals. We plan to explore this
fascinating topic in near future.

Second, it is not our goal to do a direct comparison of performance of DCQCN and
TIMELY.  Such comparison makes little sense, since both protocols offer several
tuning knobs, and given a specific scenario, either protocol can be made to
perform as well as the other. Instead, we focus on the core behavior the two
protocols to obtain broader insights.  Our key contributions, and findings are
summarized as follows.

\para{DCQCN:} $(i)$ We extend the fluid model proposed in~\cite{dcqcn}, to show
that DCQCN has a unique fixed point, where flows converge to their fair share.
Using a discrete model, we also derive that the convergence speed is exponential.
$(ii)$ We show that DCQCN is stable around this fixed point, as long as the
feedback latency is low. The relationship between stability and the number of
competing flows is, strangely, non-monotonic, which is very different from TCP's
behavior~\cite{misra:TAC2002}. DCQCN is stable for both very small, and very
large number of flows, and tends to be unstable in between; especially if the
feedback latency is high.

\para{TIMELY:} $(i)$ We develop a fluid model for TIMELY, and validate it using
simulations. The model reveals that TIMELY can have infinite fixed points,
resulting in arbitrary unfairness.  $(ii)$ We propose a simple remedy, called Patched TIMELY, and show
that the modified version is stable and exponentially converges to a unique fixed 
point. 

\para{ECN or Delay:} $(i)$ We compare the flow completion time of TIMELY and 
DCQCN running the same traffic traces. DCQCN outperforms TIMELY
due to better fairness, stability and fundamentally it is able to
control the queue length better than TIMELY. Patched TIMELY closes the gap, but 
still cannot match DCQCN performance.
$(ii)$ Based on lessons learnt from analysis of DCQCN and TIMELY,
we conclude that ECN is a better signal for conveying congestion information.
One reason for this is that modern shared buffer switches mark packets with ECN
{\em at egress}, effectively decoupling the queuing delay from feedback delay.
This improves system stability. Second, we present a \emph{fundamental} result:
for a distributed protocol that uses only delay as the feedback signal, you can
achieve either fairness or a guaranteed delay, but not both simultaneously. For
an ECN-based protocol you can achieve both by using a
PI~\cite{hollot2001designing}-like marking scheme. This is not possible with
delay-based congestion control. Finally, ECN signal is more resilient to 
jitter on the backward path as it only introduces delay in the
feedback, whereas for delay based protocols jitter introduces delay
\emph{and} noise in the feedback signal.

%% \para{General results:} $(i)$ We show that for both variants of the protocols
%% studied, the fixed point of the queue length (and hence feedback delay)
%% increases with the number of flows $(ii)$ Higher delays are undesirable both
%% from an application perspective as well as difficulty in controlling the loop.
%% To fix that, we show that using integral control, e.g. the
%% PI~\cite{hollot2001designing}-like marking scheme, we can make the queue length
%% (and hence delay) independent of the number of contending flows. $(iii)$ In
%% controlling the queue length to a fixed point, we derive a \emph{fundamental}
%% result for congestion control: for a distributed protocol that uses only delay
%% as the feedback signal, you can achieve either fairness or a guaranteed delay,
%% but not both simultaneously. For a protocol that employs ECN marking based
%% network feedback, you can achieve both. $(iv)$ We discuss the pros and cons of
%% the use of hardware rate limiters (DCQCN uses them, TIMELY does not).


%%% Local Variables:
%%% mode: latex
%%% TeX-master: "main"
%%% End:
